\documentclass[10pt]{report}

\title{SELMA}
\author{ 
Vonk, J\\
s0132778\\
Matenweg 75-201\\
\\
Florisson, M\\
s000000\\
Box Calslaan xx-30}

\usepackage{listings}
\usepackage{verbatim}

\begin{document}

\begin{titlepage}
\maketitle 
\end{titlepage}
\tableofcontents

\section{Inleiding}
Korte beschrijving van de practicumopdracht.

\section{Beknopte beschrijving}
van de programmeertaal (maximaal een A4-tje).

\section{Problemen en oplossingen}
uitleg over de wijze waarop je de problemen die je bent tegenge-
komen bij het maken van de opdracht hebt opgelost (maximaal twee A4-tjes).

\section{Syntax, context-beperkingen en semantiek}
van de taal met waar nodig nadere uitleg over de
betekenis. Geef de beschrijving bij voorkeur in dezelfde terminologie als die gebruikt is bij
de beschrijving van Triangle in Watt \& Brown (hoofdstuk 1 en appendix B).

\subsection{Lexer}
Om de code te kunnen parsen zal deze eerst door de lexer moeten gaan. Hier definieren wij een aantal 

\begin{lstlisting}[frame=single]
	Tokens			Keywords
	
COLON		':';	PRINT		'print';
SEMICOLON	';';	READ		'read';
LPAREN		'(';	VAR		'var';
RPAREN		')';	CONST		'const';
LCURLY		'{';	INT		'integer';
RCURLY		'}';	BOOL		'boolean';
COMMA		',';	CHAR		'character';
EQ		'=';	BEGIN		'begin';
APOSTROPHE	''';	END		'end.';
			IF		'if';
UMIN;			THEN		'then';
UPLUS;			ELSE		'else';
COMPOUND;		FI		'fi';
			WHILE		'while';
	Operators	DO		'do';
			OD		'od';
NOT		'!';	PROC		'procedure';
MULT		'*';	FUNC		'function';
DIV		'/';
MOD		'%';
PLUS		'+';
MINUS		'-';
RELS		'<';
RELSE		'<=';
RELGE		'>=';
RELG		'>';
RELE		'==';
RELNE		'<>';
AND		'&&';
OR		'||';
BECOMES		':=';
\end{lstlisting}



\section{Vertaalregels}
voor de taal, d.w.z. de transformaties waaruit blijkt op welke wijze een opeen-
volging van symbolen die voldoet aan een produktieregel wordt omgezet in een opeenvol-
ging van TAM-instructies. Vertaalregels zijn de ‘code templates’ van hoofdstuk 7 van Watt
\& Brown.

\section{Beschrijving van Java programmatuur}
Beknopte bespreking van de extra Java klassen die
u gedefinieerd heeft voor uw compiler (b.v. symbol table management, type checking, code
generatie, error handling, etc.). Geef ook aan welke informatie in de AST-nodes opgeslagen
wordt.

\section{Testplan en -resultaten}
Bespreking van de ‘correctheids-tests’ aan de hand van de criteria
zoals deze zijn beschreven in het §A.5 van deze appendix. Aan de hand van deze criteria moet
een verzameling test-programma’s in het taal geschreven worden die de juiste werking van de
vertaler en interpreter controleren. Tot deze test-set behoren behalve correcte programma’s
die de verschillende taalconstructies testen, ook programma’s met syntactische, semantische
en run-time fouten.
Alle uitgevoerde tests moeten op de CD-R aanwezig zijn; van een testprogramma moet de
uitvoer in de appendix opgenomen worden (zie onder).

\section{Conclusies}


\section{Appendix}

\subsection{ANTLR Lexer specificatie}
Specificatie van de invoer voor de ANTLR scanner generator,
d.w.z. de token-definities van het taaltje.

\subsection{ANTLR Parser specificatie}
Specificatie van de invoer voor de parser generator, d.w.z. de
structuur van de taal en de wijze waarop de AST gegenereerd wordt

\subsection{Alle ANTLR TreeParser specificaties}
Waarschijnlijk zult u (tenminste) twee tree parsers
gebruiken: een context checker en een code generator.

\subsection{Invoer- en uitvoer van een uitgebreid testprogramma}
Van een correct en uitgebreid test-
programma (met daarin alle features van uw programmeertaal) moet worden bijgevoegd: de
listing van het oorspronkelijk programma, de listing van de gegenereerde TAM-code (be-
standsnaam met extensie .tam) en een of meer executie voorbeelden met in- en uitvoer
waaruit de juiste werking van de gegenereerde code blijkt.

\end{document}